\begin{frame}

  \frametitle{Reproducible Research: A Motivation}

  \begin{center}
    \pgfimage[width=0.48\textwidth]{graphics/equations}%
    \hspace{1em}%
    \pgfimage[width=0.48\textwidth]{graphics/plots}%
    \vspace{1em}
  \end{center}

\end{frame}


\begin{frame}

  \frametitle{You Ask the Author for Clues...}

  You get:

  \only<2->{
    \vspace{1em}

    \begin{center}
      \includegraphics[height=0.6\textheight]{graphics/short-question-mark}
    \end{center}
  }

\end{frame}


\begin{frame}
  \frametitle{Let's interact...}

  \only<1>{
    \begin{center}
    \pgfimage[width=0.40\textwidth]{graphics/miracle}%
      \end{center}
    \hspace{1em}

    \textit{How many times?}
  }%
  \only<2>{
    \begin{center}
    \pgfimage[width=0.40\textwidth]{graphics/miracle}%
      \end{center}
    \hspace{1em}

    \textit{Crossed a publication and openly decided to \textbf{ignore it
      because it would be too hard to apply} those doubtful results on your
        research?}
  }%
  \only<3>{
    \textit{Worked day and night to \textbf{incorporate some results} on your
      own work but:}

    \begin{itemize}

    \item There were \textbf{untold parameters} that needed adjustment
      and you couldn't get hold of them?

      \item Realized the proposed algorithm \textbf{worked only on the
        specific data} shown at the original paper?

        \item Realized that something did \textbf{not quite add up} in the
        end?

        \end{itemize}
  }%
  \only<4>{
    \begin{center}
    \pgfimage[width=0.6\textwidth]{graphics/phd010708s}%
      \end{center}
    \vspace{2em}

    \textit{Had \textbf{to take over} the work from another colleague that left
    and had to start from scratch - months into programming to make things work
    again?}
  }%
  \only<5>{
    \begin{center}
    \pgfimage[width=0.6\textwidth]{graphics/phd010708s}%
      \end{center}
    \vspace{2em}

    \textit{Would have liked to \textbf{replay to someone about your work},
      but you couldn't really remember all details when you first made it
        work? Or you \textbf{could not make it work} at all?}
  }

\end{frame}


\frame{

  \begin{quotation}
    An article about computational science in a scientific publication is
    \textbf{not the scholarship} itself, it is merely \textbf{advertising} of
    the scholarship. The actual scholarship is the complete software
    development environment and the complete set of instructions which
    generated the figures.
  \end{quotation}

  \begin{flushright}
    \textit{D. Donoho, 2010\footnotemark}
  \end{flushright}

  \footnotetext[1]{\url{http://biostatistics.oxfordjournals.org/content/11/3/385.long}}
}


\frame{

  \frametitle{Enter ``Reproducible Research" (RR)\footnotemark}

  One term that aggregates work comprising of:

    \begin{itemize}
  \item a \textbf{paper}, that describe your work in all relevant details
    \item \textbf{code} to reproduce all results
    \item \textbf{data} required to reproduce the results
    \item \textbf{instructions}, on how to apply the \textit{code} on the
    \textit{data} to replicate the results on the \textit{paper}.
    \end{itemize}

    \footnotetext[2]{\url{http://reproducibleresearch.net}}

}

\frame{

  \frametitle{Levels of Reproducibility\footnotemark}

  With respect to an independent researcher (reader):

  \begin{enumerate}
      \setcounter{enumi}{-1}
    \item Irreproducible
    \item Cannot seem to reproduce
    \item Reproducible, with extreme effort ($>$ 1 month)
    \item Reproducible, with considerable effort ($>$ 1 week)
    \item Easily reproducible ($\sim$ 15 min.), but requires proprietary
      software (e.g. Matlab)
    \item \textbf{Easily reproducible ($\sim$ 15 min.), only free software}
    \only<2->{
    \item \textcolor{blue}{\textbf{Easily reproducible ($\sim$ 1 min.), using
          only a web browser}}
    }
  \end{enumerate}

  \footnotetext[4]{\textit{Reproducible Research in Signal Processing: What,
    why and how}, Vandewalle, Kovacevic and Vetterli, 2012}
}


%\frame{

%  \frametitle{Pipeline}

%  \begin{center}
%    \pgfimage[interpolate=true,width=0.9\textwidth]{graphics/rr-pipeline}
%  \end{center}

%}


%\frame{

%  \frametitle{Why?}

%  \textit{Finally, writing and distributing code and data takes time...}

%  \begin{center}
%    \pgfimage[interpolate=true,width=0.5\textwidth]{graphics/rr-in-practice}
%  \end{center}

%}

\frame{
  \frametitle{Why?}

  Boost your research \textbf{impact (visibility)}:

    \begin{itemize}

  \item \textbf{Lower entrance barrier} to your publications

    \item The current number of RR papers is \textbf{rather small} - you
    have a clear chance to stand out today:

    \begin{itemize}
  \item Only \textbf{10\% of TIP} papers provide source
    code\footnotemark.
    \end{itemize}

  \item Statistically, your work is \textbf{more valuable} if it is RR:

    \begin{itemize}

  \item \textbf{13 out of the top 15 most cited} articles in TPAMI or
    TIP provide (at least) source code

    \item The average number of citations for papers that provide
    source-code in TIP is \textbf{7 fold} that of papers that don't.

    \end{itemize}
  \end{itemize}

  \footnotetext[5]{\textit{Code Sharing is Associated with Research Impact in
    Image Processing}, Patrick Vandewalle, 2012}
}



\frame{
  \frametitle{In practice}

  \begin{center}
  \textit{Organize yourself so you are \textbf{always} doing RR}
  \end{center}

  Work as a team to:

    \begin{itemize}
      \item Organize \textbf{basic tools} so that you have a library that is
        documented and re-usable by all

      \item Organize \textbf{the data} so it is easy to replay analysis
        protocols by all

      \item Write and distribute \textbf{applications} that use your basic
        tools and data to \textbf{generate interesting results}.

    \end{itemize}

Benefits:

  \begin{itemize}

  \item Research is always kept reproducible

    \item People help each other in case of problems

    \item New colleagues can start (nearly) immediately to produce
    high-quality results.

    \end{itemize}
}


\frame{

  \frametitle{At our group:}

  We handle \textbf{Level 5} RR (free software, easy deployment) in 3 ways:

    \begin{itemize}

  \item Develop and maintain a basic set of tools that we all share:
    database protocol APIs, signal (image, audio) processing, machine
    learning - \textbf{Bob}

  \item Provide most of our \textbf{databases publicly}, free of charge

    \item Provide end-user \textbf{applications} wrapped in an \textbf{easy
      to deploy} packaging system that users (potential citers) can download,
    install and extend - \textbf{Bob packages}.

      \end{itemize}

}



